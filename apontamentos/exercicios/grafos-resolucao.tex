\documentclass[12pt,a4paper]{article}
\usepackage[portuges]{babel}
\usepackage{graphicx}
\usepackage{amsmath}
\usepackage{minted}
\usepackage[twoside,verbose,body={16cm,24cm},
left=25mm,top=20mm]{geometry}

\title{Algoritmos sobre Grafos\\ Resolução de exercícios de exames}
\author{Eduardo Freitas Fernandes}
\date{2025}

\setminted{
	frame=single,
	tabsize=4,
	breaklines=true
}

\begin{document}
	
\maketitle
	
\noindent \textbf{Exercício 1}

\begin{minted}{c}
\end{minted}

\begin{minted}{c}
\end{minted}

\begin{minted}{c}
\end{minted}

\begin{minted}{c}
\end{minted}

\noindent \textbf{Exercício 2}

\begin{minted}{c}
\end{minted}

\noindent \textbf{Exercício 3}

\begin{minted}{c}
\end{minted}

\noindent \textbf{Exercício 4}

\begin{minted}{c}
void graphComp (Graph g, graph r) {

}
\end{minted}

\noindent \textbf{Exercício 5}

\begin{minted}{c}
\end{minted}

\begin{minted}{c}
\end{minted}

\noindent \textbf{Exercício 6}

\begin{minted}{c}
\end{minted}

\noindent \textbf{Exercício 7}

\begin{minted}{c}
\end{minted}

\noindent \textbf{Exercício 8}

\begin{minted}{c}
\end{minted}

\noindent \textbf{Exercício 9}

\begin{minted}{c}
\end{minted}

\noindent \textbf{Exercício 10}

\begin{minted}{c}
\end{minted}

\noindent \textbf{Exercício 11}

\begin{minted}{c}
\end{minted}

\noindent \textbf{Exercício 12}

\begin{minted}{c}
\end{minted}

\noindent \textbf{Exercício 13}

\begin{minted}{c}
\end{minted}

\noindent \textbf{Exercício 14}

\begin{minted}{c}
\end{minted}





\end{document}