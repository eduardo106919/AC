\documentclass[a4paper,11pt]{article}
\usepackage[portuguese]{babel}
\usepackage{graphicx}
\usepackage{amsmath}
\usepackage{enumitem}
\usepackage{mdframed}

\title{Introdução à Análise de Complexidade\\ Resolução de exercícios de testes}
\author{Eduardo Freitas Fernandes}
\date{2025}


\begin{document}
	
	\maketitle
	
	\noindent \textbf{Exercício 1}\\
	
	\noindent Nesta análise, contamos o número de acessos ao array \texttt{v}:
	\begin{itemize}
		\item \textbf{melhor caso}: array estritamente crescente.
		\[ T_{melhor}(N) = \sum_{i=0}^{N-1} 1 = \Omega(N) \]
		\item \textbf{pior caso}: array descrescente.
		\[
		T_{pior}(N) = \sum_{i=0}^{N-1} 1 + \sum_{i=0}^{N-1} \sum_{j=0}^{i-1} 1 = N + \sum_{i=0}^{N-1} i = N + \frac{N \times (N-1)}{2} = O(N^2)
		\]
	\end{itemize}
	
	\noindent Nesta análise, contamos o número de acessos ao array \texttt{a}:
	\[
	T(N) = \sum_{i=0}^{N-1} \sum_{j=0}^{i-1} 2 = \sum_{i=0}^{N-1} 2 \times i = 2 \times \frac{N \times (N-1)}{2} = \Theta(N^2)
	\]
	
	
	\noindent \textbf{Exercício 2}\\
	\[
		T(N) = 
		\begin{cases}
			0 & N \leq 0 \\
			T_{processa}(N) + 2 \times T(N/2) & N > 0
		\end{cases}
		\quad = \quad
		\begin{cases}
			0 & N \leq 0 \\
			N + 2 \times T(N/2) & N > 0
		\end{cases}
	\]
	\[
		= \sum_{i=1}^{1+\log_2(N)} N = N \times (1 + \log_2(N)) = \Theta(N \times \log_2(N))
	\]
	
	
	\newpage
	
	\noindent \textbf{Exercício 3}\\
	
	\noindent Nesta análise, contamos o número de execuções da função \texttt{swap()}:
	\begin{itemize}
		\item \textbf{melhor caso}: array crescente.
		\[
		T_{bubble}(N) = \Omega(1), \quad T_{bsort}(N) = \Omega(1)
		\]
		\item \textbf{pior caso}: array estritamente decrescente.
		\[
		T_{bubble}(N) = O(N), \quad T_{bsort}(N) = \sum_{i=0}^{N} N = O(N^2)
		\]
	\end{itemize}
	
	\noindent Nesta análise, contamos o número de comparações feitas entre elementos do array \texttt{a} (i.e: \texttt{(a[i] < a[i-1])})
	\begin{itemize}
		\item \textbf{melhor caso}: array crescente, \texttt{bubble()} executa uma vez, e \texttt{bsort()} termina o ciclo.
		\[
		T_{bsort}(N) = \sum_{i=0}^{N-1} 1 = \Omega(N)
		\]
		\item \textbf{pior caso}: array estritamente decrescente, o ciclo em \texttt{bsort()} executa \texttt{N} vezes.
		\[
		T_{bsort}(N) = \sum_{i=0}^{N-1} N = O(N^2)
		\]
	\end{itemize}
	
	
	\noindent \textbf{Exercício 4}\\
	
	\noindent Nesta análise, contamos o número de vezes que a operação \texttt{k*=i} é executada:
	\[
	T_{factorial}(N) = \Theta(N), \quad T_{factoriais}(N) = \sum_{i=0}^{N-1} N = \Theta(N^2)
	\]
	
	
	\noindent \textbf{Exercício 5}\\
	
	
	\noindent \textbf{Exercício 6}\\
	
	\noindent Nesta análise, contamos o número de vezes que a condição \texttt{B[j] <= A[i]} é avaliada:
	\begin{itemize}
		\item \textbf{melhor caso}: 
		\[
		T(N) = \sum_{i=0}^{N-1} 1 = \Omega(N)
		\]
		\item \textbf{pior caso}: 
		\[
		T(N) = \sum_{i=0}^{N-1} \sum_{j=0}^{i-1} 1 = \sum_{i=0}^{N-1} i = \frac{N \times (N-1)}{2} = O(N^2)
		\]
	\end{itemize}
	
	
	\noindent \textbf{Exercício 7}\\
	\[
		T(N) = 
		\begin{cases}
			0 & N \leq 0 \\
			1 + 2 \times T(N/2) & N > 0
		\end{cases}
		= \sum_{i=0}^{\log_2(N)} 2^i = \frac{2^{1+ \log_2 N} - 1}{2-1} = \Theta(\log_2 N)
	\]


	\noindent \textbf{Exercício 8}\\
	
	\noindent Nesta análise, iremos avaliar o pior caso, em que \texttt{k = N}:
	\[
	T(N) = 
	\begin{cases}
		0 & N \leq 0 \\
		N + T(N-1) & N > 0
	\end{cases}
	= \sum_{i=1}^{N} i = \frac{N \times (N+1)}{2} = O(N^2)
	\]
	
	
	\noindent \textbf{Exercício 9}
\begin{mdframed}
\begin{verbatim}
void shift(int u[], int n, int k) {
    int storage[N];
    int i;
    // calcular novas posicoes
    for (i = 0; i < N; i++)
        storage[(i + k) % n] = u[i];
    // copiar para o array original
    for (i = 0; i < N; i++)
        u[i] = storage[i];
}
\end{verbatim}
\end{mdframed}	
	
	
	\noindent \textbf{Exercício 10}\\
	
	\noindent Nesta análise, iremos contar o número de vezes que o array \texttt{bits} é alterado (i.e: \texttt{bits[n] = ...}):
	\begin{itemize}
		\item \textbf{melhor caso}: bit menos significativo (à direita) a zero.
		\[ T(N) = \Omega(1) \]
		\item \textbf{pior caso}: todos os bits a 1, ou o mais significativo a 0, e os restantes a 1.
		\[ T(N) = \sum_{i=0}^{N-1} 1 = O(N) \]
	\end{itemize}
	
	\begin{center}
		\begin{tabular}{|c|c|c|c|}
			\hline
			i & input & output & $ c_i $ \\
			\hline
			\hline
			1 & \fbox{0} \fbox{0} \fbox{0} \fbox{0} & \fbox{0} \fbox{0} \fbox{0} \fbox{1} & 1 = 1 \\
			2 & \fbox{0} \fbox{0} \fbox{0} \fbox{1} & \fbox{0} \fbox{0} \fbox{1} \fbox{0} & 2 = 1 + 1 \\
			3 & \fbox{0} \fbox{0} \fbox{1} \fbox{0} & \fbox{0} \fbox{0} \fbox{1} \fbox{1} & 1 = 1 \\
			4 & \fbox{0} \fbox{0} \fbox{1} \fbox{1} & \fbox{0} \fbox{1} \fbox{0} \fbox{0} & 3 = 1 + 1 + 1 \\
			5 & \fbox{0} \fbox{1} \fbox{0} \fbox{0} & \fbox{0} \fbox{1} \fbox{0} \fbox{1} & 1 = 1 \\
			6 & \fbox{0} \fbox{1} \fbox{0} \fbox{1} & \fbox{0} \fbox{1} \fbox{1} \fbox{0} & 2 = 1 + 1 \\
			7 & \fbox{0} \fbox{1} \fbox{1} \fbox{0} & \fbox{0} \fbox{1} \fbox{1} \fbox{1} & 1 = 1 \\
			8 & \fbox{0} \fbox{1} \fbox{1} \fbox{1} & \fbox{1} \fbox{0} \fbox{0} \fbox{0} & 4 = 1 + 1 + 1 + 1 \\
			9 & \fbox{1} \fbox{0} \fbox{0} \fbox{0} & \fbox{1} \fbox{0} \fbox{0} \fbox{1} & 1 = 1 \\
			10 & \fbox{1} \fbox{0} \fbox{0} \fbox{1} & \fbox{1} \fbox{0} \fbox{1} \fbox{0} & 2 = 1 + 1 \\
			\hline
		\end{tabular}
	\end{center}
	
	\noindent O que pretendemos calcular é a soma dos elementos da última coluna $ \sum c_i $ para depois calcular a sua média. Neste caso, podemos ver que:
	\begin{itemize}
		\item todas as linhas têm a parcela 1
		\item metade das linhas têm também a parcela +1
		\item um quarto das linhas têm também mais uma parcela +1
		\item ...
	\end{itemize}
	
	\noindent Generalizando, a soma dos valores da última coluna pode ser calculada por:
	\[
	C_N = N + \frac{N}{2} + \frac{N}{4} + ... = \sum_{i=0}^{\log_2 N} \frac{N}{2^i} = N \times (2 - \frac{1}{2^{\log_2 N}}) = 2 \times N - 1
	\]
	
	\noindent Para sabermos o custo amortizado de cada operação, temos que dividir este custo pelo número de operações em causa:
	\[
	\hat{c}_i = \frac{C_N}{N} = \frac{2 \times N - 1}{N} = 2 - \frac{1}{N} = \Theta(1)
	\]
	
	
	\noindent \textbf{Exercício 11}\\
	
	
	
	
	
	
	
\end{document}