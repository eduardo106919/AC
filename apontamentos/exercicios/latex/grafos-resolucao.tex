\documentclass[12pt,a4paper]{article}
\usepackage[portuges]{babel}
\usepackage{graphicx}
\usepackage{amsmath}
\usepackage{minted}
\usepackage[twoside,verbose,body={16cm,24cm},
left=25mm,top=20mm]{geometry}

\title{Algoritmos sobre Grafos\\ Resolução de exercícios de exames}
\author{Eduardo Freitas Fernandes}
\date{2025}

\setminted{
	frame=single,
	tabsize=4,
	breaklines=true
}

\begin{document}
	
\maketitle
	
\noindent \textbf{Exercício 1}

\begin{minted}{c}
typedef enum { Vermelho, Amarelo, Azul, Verde } Cor;
	
struct node {
	int dest;
	Cor linha;
	struct node *next;
};
	
typedef struct grafo {
	int capacidade;
	int tamanho;
	char ** estacoes;
	struct node **entradas;
} Grafo;	
\end{minted}

\begin{minted}{c}
void initGrafo(Grafo * g) {
	g->capacidade = 10;
	g->tamanho = 0;
	
	g->estacoes = malloc(g->capacidade * sizeof(char *));
	g->entradas = malloc(g->capacidade * sizeof(struct node *));
	
	for (int i = 0; i < g->capacidade; i++) {
		g->estacoes[i] = NULL;
		g->entradas[i] = NULL;
	}
}

void addEstacao(Grafo * g, const char * estacao) {
	// array cheio
	if (g->tamanho == g->capacidade) {
		g->capacidade *= 2;
		g->estacoes = realloc(g->estacoes, g->capacidade * sizeof(char *));
		g->entradas = realloc(g->entradas, g->capacidade * sizeof(struct node *));
	}
	
	g->estacoes[g->tamanho++] = strdup(estacao);
}

void addLinha(Grafo * g, const char * origem, const char * destino, Cor linha) {
	int or, dest;
	// procurar indice de origem
	for (or = 0; or < g->tamanho && strcmp(g->estacoes[or], origem); or++);
	
	// adicionar origem, caso não exista
	if (or == g->tamanho) {
		addEstacao(g, origem);
		or = g->tamanho - 1;
	}
	
	// procurar indice de destino
	for (dest = 0; dest < g->tamanho && strcmp(g->estacoes[dest], origem); dest++);
	
	// adicionar destino, caso não exista
	if (dest == g->tamanho) {
		addEstacao(g, destino);
		dest = g->tamanho - 1;
	}
	
	struct node ** temp = &(g->entradas[or]);
	// adicionar ordenadamente
	while (*temp != NULL && (*temp)->dest < dest)
		temp = &((*temp)->next);
	
	// adicionar aresta
	struct node * aux = malloc(sizeof(struct node));
	aux->next = *temp;
	*temp = aux;
}
\end{minted}

\begin{minted}{c}
int passaPorEstacao(Grafo * g, const char * estacao, int visitados[]) {
}
\end{minted}

\begin{minted}{c}
int existeLigacao(Grafo * g, const char * origem, const char * destino, Cor linha) {
}
\end{minted}

\noindent \textbf{Exercício 2}

\begin{minted}{c}
\end{minted}

\noindent \textbf{Exercício 3}

\begin{minted}{c}
\end{minted}

\noindent \textbf{Exercício 4}

\begin{minted}{c}
void graphComp (Graph g, graph r) {

}
\end{minted}

\noindent \textbf{Exercício 5}

\begin{minted}{c}
\end{minted}

\begin{minted}{c}
\end{minted}

\noindent \textbf{Exercício 6}

\begin{minted}{c}
\end{minted}

\noindent \textbf{Exercício 7}

\begin{minted}{c}
\end{minted}

\noindent \textbf{Exercício 8}

\begin{minted}{c}
\end{minted}

\noindent \textbf{Exercício 9}

\begin{minted}{c}
\end{minted}

\noindent \textbf{Exercício 10}

\begin{minted}{c}
\end{minted}

\noindent \textbf{Exercício 11}

\begin{minted}{c}
\end{minted}

\noindent \textbf{Exercício 12}

\begin{minted}{c}
\end{minted}

\noindent \textbf{Exercício 13}

\begin{minted}{c}
\end{minted}

\noindent \textbf{Exercício 14}

\begin{minted}{c}
\end{minted}





\end{document}