\documentclass{article}
\usepackage{graphicx} % Required for inserting images
\usepackage{listings} % Import the package
\usepackage{xcolor} % For syntax highlighting colors
\usepackage{amsmath}
\usepackage{float}
\usepackage{array}
\usepackage[a4paper, left=4cm, right=4cm, top=4cm, bottom=3cm]{geometry}
\usepackage{enumitem}


\setlength{\topskip}{0pt}
\setlength{\voffset}{-1cm}

\lstdefinestyle{mystyle}{
    language=C,
    basicstyle=\ttfamily\footnotesize,
    keywordstyle=\color{blue},
    commentstyle=\color{gray},
    stringstyle=\color{red},
    numbers=left,
    numberstyle=\tiny,
    stepnumber=1,
    breaklines=true,
    frame=single
}


\title{Ficha 1 \\ Resolução}
\author{Eduardo Freitas Fernandes}
\date{2025}

\begin{document}

\maketitle



\section{Especificações}

\noindent \textbf{1.} Descreva o que faz cada uma das seguintes funções:

\begin{enumerate}[label=\alph*)]
    \item determina o máximo entre \texttt{x} e \texttt{y}
    \item determina um divisor comum de \texttt{x} e \texttt{y}
    \item determina um múltiplo comum de \texttt{x} e \texttt{y}
    \item determina o índice do menor elemento do array \texttt{a[]}
    \item determina um minorante do array \texttt{a[]}
    \item determina o menor elemento do array \texttt{a[]}
    \item determina a posição de \texttt{x} no array, caso não exista, retorna -1
\end{enumerate}

\noindent \textbf{2.} Escreva as pré e pós-condições para as seguintes funções.

\begin{lstlisting}[style=mystyle]
int prod (int x, int y) {
    // pre: True
    ...
    // pos: r == x*y
}
\end{lstlisting}

\begin{lstlisting}[style=mystyle]
int mdc (int x, int y) {
    // pre: True
    ...
    // pos: (x % r == 0 && y % r == 0) &&
    //      (forall_{x % p == 0 && y % p == 0} p <= r)
}
\end{lstlisting}

\begin{lstlisting}[style=mystyle]
int sum (int v[], int N) {
    // pre: N > 0
    ...
    // pos: r == sum_{0 <= i < N} a[i]
}
\end{lstlisting}

\begin{lstlisting}[style=mystyle]
int maxPOrd (int v[], int N) {
    // pre: N > 0
    ...
    // pos: (0 < r <= N) && (forall_{0 < i < r} v[i-1] <= v[i]) &&
    //      (forall_{forall_{0 < i < p} v[i - 1] <= v[i]} p <= r)
}
\end{lstlisting}

\begin{lstlisting}[style=mystyle]
int isSorted (int v[], int N) {
    // pre: N > 0
    ...
    // pos: (forall_{0 < i < N} v[i-1] <= v[i] && r > 0) ||
    //      (r == 0 && exists_{0 < i < N} v[i-1] > v[i])
}
\end{lstlisting}


\section{Correção}
\noindent \textbf{1.} Para cada um dos seguintes triplos de Hoare, apresente um contra-exemplo que mostre a sua \textbf{não} validade.

\begin{enumerate}[label=\alph*)]
    \item \texttt{x = 0, y = -5}
    \item \texttt{x = 3, y = 2}
    \item \texttt{x = 0, y = 0}
    \item \texttt{x = 0, y = 0}
    \item \texttt{x = 1, y = -10}
\end{enumerate}

\noindent \textbf{2.} Modifique a pré-condição de cada um dos triplos de Hoare da alínea anterior de forma a obter um triplo válido.

\begin{lstlisting}[style=mystyle]
{y >= 0}
r = x+y;
{r >= x}
\end{lstlisting}

\begin{lstlisting}[style=mystyle]
{x == y}
x = x+y; y = x-y; x = x-y;
{x == y}
\end{lstlisting}

\begin{lstlisting}[style=mystyle]
{x != y}
x = x+y; y = x-y; x = x-y;
{x != y}
\end{lstlisting}

\begin{lstlisting}[style=mystyle]
{x != y}
if (x > y) r = x-y; else r = y-x;
{r > 0}
\end{lstlisting}

\begin{lstlisting}[style=mystyle]
{y > 0}
while (x > 0) { y = y+1; x = x-1;}
{y > x}
\end{lstlisting}

\noindent \textbf{3.} Para cada uma das 4 primeiras alíneas do exercício anterior, mostre que a alteração que propôs é de facto um triplo válido.

\vspace{0.5cm}
(a)
\[
\dfrac{ 
    \dfrac{\{ y \geq 0 \} \Rightarrow x + y \geq x}
         {\{ y \geq 0 \} \Rightarrow \{ r \geq x[r \backslash x+y] \} }
}
     {\{ y \geq 0 \} r = x + y \{ r \geq 0 \} }
\]

\vspace{0.5cm}
(b)
\[
\dfrac{
    \dfrac{
        \dfrac{
            \dfrac{\{ x == y \} \Rightarrow \{ y == x \}}
                  {\{ x == y \} \Rightarrow \{ y == x-y[x \backslash x+y] \}}
              }
              {\{ x == y \} x = x+y \{ y == x-y \}}
          }
          {\{ x == y \} x = x+y \{ R \}}
    \dfrac{
        \dfrac{
            \dfrac{
                \dfrac{\{ R \} \Rightarrow \{ y == x-y \}}
                      {\{ R \} \Rightarrow \{ x-y == y[y \backslash x-y] \}}
                  }
                  {\{ R \} y = x-y \{ x-y == y \}}
              }
              {\{ R \} y = x-y \{ K \}}
        \dfrac{
            \dfrac{\{ K \} \Rightarrow \{ x-y == y\}}
                  {\{ K \} \Rightarrow \{ x == y[x \backslash x-y] \}}
              }
              {\{ K \} x = x-y \{ x == y \}}
          }
          {\{ R \} y = x-y; x = x-y \{ x == y \}}
      }
      {\{ x == y \} x = x+y; y = x-y; x = x-y \{ x == y \}}
\]

\vspace{0.5cm}
(c) Semelhante à alínea (b)

\vspace{0.5cm}
(d)
\[
\dfrac{
    \dfrac{
        \dfrac{
            \dfrac{\{ x > y \} \Rightarrow \{ x-y > 0 \}}
                  {\{ x > y \} \Rightarrow \{ r > 0[r \backslash x-y] \}}
              }
              {\{ x > y \} r = x-y \{ r > 0 \}}
          }
          {\{ x \neq y \wedge x > y \} r = x-y \{ r > 0 \}}
    \dfrac{
        \dfrac{
            \dfrac{\{ x < y \} \Rightarrow \{ y-x > 0 \}}
                  {\{ x < y \} \Rightarrow \{ r > 0[r \backslash y-x] \}}
              }
              {\{ x < y \} r = y-x \{ r > 0 \}}
          }
          {\{ x \neq y \wedge x \leq y \} r = y-x \{ r > 0 \}}
      }
      {\{ x \neq y \} \ \mathbf{if} \ (x>y) \ r = x-y \ \mathbf{else} \ x = y-x \{ r > 0 \}}
\]


\section{Invariantes}

\textbf{1.} Considere as seguintes implementações de uma função que calcula o produto de dois números.

\noindent Para cada um dos predicados, indique se são verdadeiros no início (Init) e preservados pelos ciclos destas duas funções.

\begin{table}[H]
    \centering
    \renewcommand{\arraystretch}{1.5}
    \begin{tabular}{|c|c|c|c|c|}
        \hline
        \textbf{Predicado} & \multicolumn{2}{c|}{\textbf{mult1}} & \multicolumn{2}{c|}{\textbf{mult2}} \\
        \cline{2-5}
        & \textbf{Init} & \textbf{Pres} & \textbf{Init} & \textbf{Pres} \\
        \hline
        $r == a \times b$ & \textcolor{red}{F} & \textcolor{red}{F} & \textcolor{red}{F} & \textcolor{red}{F} \\
        \hline
        $a \geq 0$ & \textcolor{teal}{V} & \textcolor{teal}{V} & \textcolor{teal}{V} & \textcolor{teal}{V} \\
        \hline
        $b \geq 0$ & \textcolor{red}{F} & \textcolor{red}{F} & \textcolor{red}{F} & \textcolor{red}{F} \\
        \hline
        $r \geq 0$ & \textcolor{teal}{V} & \textcolor{red}{F} & \textcolor{teal}{V} & \textcolor{red}{F} \\
        \hline
        $a == x$ & \textcolor{teal}{V} & \textcolor{red}{F} & \textcolor{teal}{V} & \textcolor{red}{F} \\
        \hline
        $a \neq 0$ & \textcolor{red}{F} & \textcolor{red}{F} & \textcolor{red}{F} & \textcolor{red}{F} \\
        \hline
        $b == y$ & \textcolor{teal}{V} & \textcolor{teal}{V} & \textcolor{teal}{V} & \textcolor{red}{F} \\
        \hline
        $a \times b == x \times y$ & \textcolor{teal}{V} & \textcolor{red}{F} & \textcolor{teal}{V} & \textcolor{red}{F} \\
        \hline
        $a \times b + r == x \times y$ & \textcolor{teal}{V} & \textcolor{teal}{V} & \textcolor{teal}{V} & \textcolor{teal}{V} \\
        \hline
    \end{tabular}
\end{table}

\noindent Apresente invariantes dos ciclos destas duas funções que lhe permitam provar a sua correção (parcial).
\\ \\
\texttt{mult1:} $a \geq 0 \quad \land \quad b == y \quad \land \quad a \times b + r == x \times y $
\\
\texttt{mult2:} $a \geq 0 \quad \land \quad a \times b + r == x \times y $
\\ \\

\noindent \textbf{2.} Para cada uma das funções seguintes, indique um \textbf{invariante} de ciclo que lhe permita provar a \textbf{correção parcial}. Em cada um dos casos, mesmo informalmente, apresente argumentos que lhe permitam demonstrar as propriedades \texttt{(inicialização, preservação e utilidade)} dos invariantes definidos.

\noindent Indique também um \textbf{variante} de ciclo que lhe permita provar a \textbf{correção total}.

\begin{enumerate}[label=\alph*)]
    \item I: $ \quad 0 \leq r \leq N \quad \land \quad i \leq N \quad \land \quad \forall_{0 \le k < i} \  v[r] \leq v[k] $
    
    V: $ \quad N - i $
    
    \item I: $ \quad i \leq N \quad \land \quad \forall_{0 \leq k < i} \  r \leq v[k] \quad \land \quad \exists_{0 \leq m < i} \  r == v[m] $
    
    V: $ \quad N - i $
    
    \item I: $ \quad 0 \leq i \leq N \quad \land \quad r == \sum_{K = 0}^{i} v[k] $
    
    V: $ \quad N - i $
    
    \item I: $ \quad a \times b + r == x^2 $
    
    V: $ \quad a $
    
    \item I: $ \quad i \leq x \quad \land \quad r == i^2 $
    
    V: $ \quad x - i $
    
    \item I: $ \quad r \leq N \quad \land \quad \forall_{0 < k < r} \  v[k-1] \leq v[k] $
    
    V: $ \quad N - r $
    
    \item I: $ \quad i \leq N \; \land \; ((p == -1 \; \land \; \forall_{0 \le k < i} \  a[k] \neq x) \; \lor \; (0 \leq p < i \; \land \; x == a[p])) $
    
    V: $ \quad N - i $
    
    \item I: $ \quad i \leq N \; \land \; ((p == -1 \; \land \; \forall_{0 \le k < i} \  a[k] < x) \; \lor \; (0 \leq p < i \; \land \; x == a[p])) $
    
    V: $ \quad N - i $
    
    \item I: $ \quad i \leq N \quad \land \quad 0 \leq s \quad \land \quad \forall_{0 \leq k < i} \  x \neq a[k] \quad \land \quad \forall_{s < j < N} \  x \neq a[j] $
    
    V: $ \quad s - i + 1 $
    
    \item I: $ \quad i \leq n + 1 \quad \land \quad (r == 0 \quad \lor \quad r == \frac{i \times (i-1)}{2}) $
    
    V: $ \quad m + 1 - i $
    
    \item I: $ \quad i \leq 0 \quad \land \quad r == \frac{n \times (n+1) - i \times (i+1)}{2} $
    
    V: $ \quad i $
    
    \item I: $ \quad \exists_k \  x == k \times y + r \quad \land \quad r \geq 0 $
    
    V: $ \quad r - y + 1 $
    
    \item I: $ \quad i \leq N \quad \land \quad r == \sum_{K=0}^{i-1} \  x^k \times coef[k] $
    
    V: $ \quad N - i $
    
    \item $ \quad i \geq 0 \quad \land \quad r == \sum_{K=0}^{N-i-1} \  x^k \times coef[i+k] $
    
    V: $ \quad i $
\end{enumerate}


\end{document}
