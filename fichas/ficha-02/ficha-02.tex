\documentclass[a4paper,11pt]{article}
\usepackage[portuguese]{babel}
\usepackage{graphicx} % Required for inserting images
\usepackage{amsmath}
\usepackage{enumitem}


\title{Algoritmos e Complexidade\\ Ficha 2\\ Resolução}

\author{Eduardo Freitas Fernandes}
\date{2025}

\begin{document}
	
	\maketitle
	
	
	\section{Contagem}
	
	\textbf{Exercício 1}\\
	
	\noindent \textbf{Função \texttt{bubbleSort()}}:\\
	
	\noindent \textbf{Comparações entre elementos do array}:
	O número de comparações entre elementos do array é igual em todos os casos, logo não existem situações em que o número de comparações mude.\\
	
	\noindent \textbf{Trocas Efetuadas}:
	\begin{itemize}
		\item Melhor caso: array ordenado por ordem crescente (condição do \texttt{if statement} é sempre falsa)
		\item Pior caso: array ordenado por ordem decrescente (condição do \texttt{if statement} é sempre verdadeira)
	\end{itemize}
	
	\noindent Número de comparações:
	\[
	\sum_{i=1}^{N-1} \sum_{j=0}^{i-1} 1 = \sum_{i=1}^{N-1} i = (N-1) \times N = \Theta(N^2)
	\]
	
	\noindent \textbf{Função \texttt{iSort()}}:\\
	
	\noindent \textbf{Comparações entre elementos do array}:
	\begin{itemize}
		\item Melhor caso: array ordenado por ordem crescente (apenas uma comparação no loop interno, de cada iteração do loop externo)
		\item Pior caso: array ordenado por ordem decrescente
	\end{itemize}
	
	\noindent Para as trocas de elementos, o melhor e pior caso são os mesmos, dado que o \texttt{swap} depende da condição de paragem do \texttt{for loop}.\\
	
	\noindent \textbf{Melhor caso}:
	\[ \sum_{i=1}^{N-1} 1 = N-1 = \Omega(N) \]
	\textbf{Pior caso}:
	\[ \sum_{i=1}^{N-1} \sum_{j=1}^{i} 1 = \sum_{i=1}^{N-1} i = (N-1) \times N = O(N^2) \]
	
	
	\noindent \textbf{Exercício 2}
	
	\noindent \textbf{Exercício 3}
	
	\[
	T(N) = ...
	\]
	
	\begin{verbatim}
		int maxSoma (int v[], int N) {
			
		}
	\end{verbatim}
	
	\[
	T(N) = ...
	\]
	
	
	\noindent \textbf{Exercício 4}
	
	\section{Definições Recursivas}
	
	\noindent \textbf{Exercício 1}\\
	
	\noindent a)
	
	\begin{figure}[h]
		\centering
		\includegraphics[width=0.7\textwidth]{imgs/2_1-a}
		%\caption{}
		\label{fig:2_1-a}
	\end{figure}
	
	\[
		T(N) = \sum_{i=0}^{N} K = (N + 1) \times K = \Theta(N)
	\]
	
	\noindent b)
	
	\begin{figure}[h]
		\centering
		\includegraphics[width=0.7\textwidth]{imgs/2_1-b}
		%\caption{}
		\label{fig:2_1-b}
	\end{figure}
	
	\[
		T(N) = \sum_{i=0}^{1 + \log_2(N)} K = (2 + \log_2(N)) \times K = \Theta(\log_2(N))
	\]
	
	\noindent c)
	
	\begin{figure}[h]
		\centering
		\includegraphics[width=0.7\textwidth]{imgs/2_1-c}
		%\caption{}
		\label{fig:2_1-c}
	\end{figure}
	
	\[
		T(N) = \sum_{i=0}^{1 + \log_2(N)} 2^i \times K = K \times (2^{\log_2(N) + 2} - 1) = K \times (4 \times N - 1) = \Theta(N)
	\]
	
	\noindent d)
	
	\begin{figure}[h]
		\centering
		\includegraphics[width=0.7\textwidth]{imgs/2_1-d}
		%\caption{}
		\label{fig:2_1-d}
	\end{figure}
	
	\[
		T(N) = K + \sum_{i=1}^{N} i = K + \frac{N \times (N + 1)}{2} = \Theta(N^2)
	\]
	
	\noindent e)
	
	\begin{figure}[h]
		\centering
		\includegraphics[width=0.7\textwidth]{imgs/2_1-e}
		%\caption{}
		\label{fig:2_1-e}
	\end{figure}
	
	\[
		T(N) = k + \sum_{i=1}^{1 + \log_2(N)} \frac{N}{2^i} = K + 2^{\log_2(N) + 1} - 1 = K + 2 \times N - 1 = \Theta(N)
	\]
	
	\noindent f)
	
	\begin{figure}[h]
		\centering
		\includegraphics[width=0.7\textwidth]{imgs/2_1-f}
		%\caption{}
		\label{fig:2_1-f}
	\end{figure}
	
	\[
		T(N) = k + \sum_{i=1}^{1 + \log_2(N)} N = K + N \times (1 + \log_2(N)) = \Theta(N \times \log_2(N))
	\]
	
	\noindent \textbf{Exercício 2}
	\[
		T(N) =
		\begin{cases}
			0 & N \leq 0 \\
			1 + N - 1 + T(N - 1) & N > 0
		\end{cases}
		\quad = \quad
		\begin{cases}
			0 & N \leq 0 \\
			N + T(N - 1) & N > 0
		\end{cases}
	\]
	\[
		= \sum_{i=1}^{N} i = \frac{N \times (N + 1)}{2} = \Theta(N^2)
	\]
	
	
	\noindent \textbf{Exercício 3}
	
	\[
		T(N) = 
		\begin{cases}
			0 & N \leq 0 \\
			1 + 2 \times T(N - 1) & N > 0
		\end{cases}
	\]
	
	\begin{figure}[h]
		\centering
		\includegraphics[width=0.7\textwidth]{imgs/2_3}
		%\caption{}
		\label{fig:2_3}
	\end{figure}
	
	\[
		T(N) = \sum_{i=0}^{N-1} 2^i = 2^N - 1 = \Theta(2^N)
	\]
	
	\noindent \textbf{Exercício 4}
	
	\[
		T(N) = 
		\begin{cases}
			1 & N \leq 1 \\
			T_{mergeH}(N) + 2 \times T(N / 2) & N > 1
		\end{cases}
		=
		\begin{cases}
			1 & N \geq 1 \\
			2 \times N + 2 \times T(N/2) & N > 1
		\end{cases}
	\]
	
	\[
		= \sum_{i=1}^{1 + \log_2(N)} 2 \times N = 2 \times N \times (1 + \log_2(N)) = \Theta(N \times \log_2(N))
	\]
	
	\noindent \textbf{Exercício 5}\\
	
	\noindent \textbf{Árvores Equilibradas}
	
	\[
		T(N) = 
		\begin{cases}
			0 & N \leq 0 \\
			1 + 2 \times T(\frac{N-1}{2}) & N > 0
		\end{cases}
		=
		\begin{cases}
			0 & N \leq 0 \\
			1 + 2 \times T(N/2) & N > 0
		\end{cases}
	\]
	\[
		= \sum_{i=0}^{1 + \log_2(N)} 2^i = 4 \times N - 1 = \Theta(N)
	\]
	
	\noindent \textbf{Árvores "Lista"}
	
	\[
	T(N) = 
	\begin{cases}
		0 & N \leq 0 \\
		1 + T(N - 1) & N > 0
	\end{cases}
	\quad = \quad \sum_{i=1}^{N} 1 = N = \Theta(N)
	\]
	
	
	
	\section{Análise de Caso Médio}
	
	\noindent \textbf{Exercício 1}
	
	\noindent \textbf{Exercício 2}
	
	\noindent \textbf{Exercício 3}
	
	\noindent \textbf{Exercício 4}
	
	\noindent \textbf{Exercício 5}
	
	\noindent \textbf{Exercício 6}
	
	
	\section{Análise Amortizada}
	
	\noindent \textbf{Exercício 1}
	
	\noindent \textbf{Exercício 2}
	
	\noindent \textbf{Exercício 3}
	
	
\end{document}